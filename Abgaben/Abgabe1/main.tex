\documentclass[11pt]{article}
\usepackage{siunitx}
\sisetup{
  locale = DE ,
  per-mode = symbol  % whether it should print "/" or "^-1"
}

\usepackage[margin=2cm]{geometry}
\usepackage{multicol}
\usepackage{nicefrac}

% \usepackage{pdflscape}
% \usepackage{pdfpages}
% \usepackage{epstopdf}
% \epstopdfDeclareGraphicsRule{.tiff}{png}{.png}{convert -density 180 #1 \OutputFile}
% \AppendGraphicsExtensions{.tiff}


\author{Team 03}
\title{Abgabe 1 Autonomes Fahren}
\begin{document}
\maketitle
\tableofcontents

\section{Masse}

    \subsection{gesamtes Auto}
    Die Waage kann nur eine Masse bis \SI{2}{\kilogram} messen, deshalb wurde wie folgt ein Gesamtgewicht von \SI{2261,13}{\gram} errechnet:
    \begin{itemize}
        \item Akku: $\SI{404,32}{\gram}$
        \item Fahrzeug: $\SI{1841,36}{\gram}$
        \item Akkuhalterung: $\SI{15,45}{\gram}$
    \end{itemize}

    \subsection{Einzelmessungen}
    Für spätere Berechnungen und zur Sicherheit wurde eine Messung der enthaltenen Einzelteile (soweit möglich) durchgeführt. Dies hat folgende Massen ergeben:
    \begin{multicols}{2}
    \begin{itemize}
        \item Einzelnes Rad: $\SI{37,35}{\gram}$
        \item 4 Räder: $\SI{149,74}{\gram}$
        \item Motor: $\SI{181,87}{\gram}$
        \item Raspberry Pi: $\SI{50,18}{\gram}$
        \item IBT\_2 (blau): $\SI{65,99}{\gram}$
        \item Verschaltung: $\SI{48,13}{\gram}$
        \item Chassis: $\SI{762,99}{\gram}$
        \item Kameraaufhängung: $\SI{147,05}{\gram}$
        \item Grundplatte für Technik: $\SI{227,32}{\gram}$
        \item Servomotor: $\SI{63,81}{\gram}$
        \item Kamera: $\SI{3,38}{\gram}$
        \item Schalter: NaN
        \item div Schrauben: $\SI{3,73}{\gram}$
        \item div Schrauben: $\SI{4,14}{\gram}$
        \item div Schrauben (Verbindung vom Chassis zur Technik): $\SI{38,11}{\gram}$
        \item IMU (beschleunigungssensor): NaN
        \item Kabel zwischen blauer Platine und Steuerungseinheit: $\SI{7,12}{\gram}$
        \item Sicherung: $\SI{34,47}{\gram}$
    \end{itemize}
    \end{multicols}

\section{Schwerpunkt}
    Der wahre Schwerpunkt war nicht zu ermitteln, dieser liegt im Inneren der Karosserie.
    Wir haben die Schwerpunktslage bezogen auf die Grundfläche auf zwei verschieden Arten ermittelt.
    Zum einen wurde das Gewicht mit Federwagen in $X$-Richtung gemessen, hierbei traten folgende Kräfte auf:
    \begin{itemize}
    \item vorne: $\SI{7,1}{\newton}$
    \item hinten: $\SI{14,5}{\newton}$
    \item Abstand zwischen den Messpunkten: $\SI{31,5}{\cm}$
    \end{itemize}
    Dies führt zu einer Schwerpunktslage von $31,5\cdot\nicefrac{7,1}{7,1+14,5} \approx 10,354$ gegenüber dem hinteren Messpunkt beziehungsweise einer Schwerpunktslage von $31,5\cdot\nicefrac{14,5}{7,1+14,5} \approx 21,146$ gegenüber dem vorderen Messpunkt.

    Weiter haben wir eine Messung mit Waage durchgeführt.
    Hierbei wurde eine Achse aufgelegt und gemessen, während die andere in Gleichgewichtslage fix gehalten wurde. Gemessen wurden folgende Werte:
    \begin{itemize}
    \item vorne: $\SI{907,4}{\gram}$
    \item hinten: $\SI{1305,3}{\gram}$
    \item Abstand zwischen den Achsen (Messpunkten): \SI{28,5}{\cm}
    \end{itemize}
    Dies führt zu einer Schwerpunktslage von $28,5\cdot\nicefrac{907,4}{907,4+1305,3} \approx 11,687$ gegenüber dem hinteren Messpunkt beziehungsweise einer Schwerpunktslage von $28,5\cdot\nicefrac{1305,3}{907,4+1305,3} \approx 17,608$ gegenüber dem vorderen Messpunkt.


    Hierbei sind wir davon ausgegangen, dass der Schwerpunkt in $Y$-Richtung (seitlich) zu vernachlässigen sei. Zwei Messungen mit Federwagen haben folgende Ergebnisse geliefert:
    \begin{itemize}
        \item links \SI{11,1}{\newton} sowie rechts \SI{9}{\newton}
        \item links \SI{10}{\newton} sowie rechts \SI{10,5}{\newton}
    \end{itemize}

    Die Unterschiede sind hier auf Messfehler zurückzuführen, im Mittel ist die Schwerpunktslage in diese Richtung zu vernachlässigen und nur wie oben beschrieben in $X$-Richtung zu betrachten.
    Auch in $X$-Richtung traten verschiedene Unterschiede auf, im Mittel lässt sich aber (wie erwartet) sagen dass sich der Schwerpunkt etwa im hinteren Drittel auf Höhe des Motors befindet.

\section{Trägheit}
[1]: file:///Users/nico.albers/Downloads/MDO3000-Oscilloscope-User-Manual.pdf


$T=\frac{2\pi}{\omega} = 2\pi\sqrt{\frac{I}{mgl}}$
* T Periodendauer
* I Trägheitsmoment
* $\omega$: Kreisfrequenz
* m Masse
* g Erdbeschleunigung
* l Länge des Pendels



\section{Radius}
maximaler Einschlagwinkel (300), 133,7m


\section{dynamischer radius}
\begin{itemize}
    \item 1. run 500
    \item 2. run 1000
    \item 3. run 2000
\end{itemize}








\end{document}
