\documentclass[12pt,final]{article}
% \usepackage{pdflscape}
% \usepackage{pdfpages}

% \usepackage{epstopdf}
% \epstopdfDeclareGraphicsRule{.tiff}{png}{.png}{convert -density 180 #1 \OutputFile}
% \AppendGraphicsExtensions{.tiff}


\author{Team 03}
\title{Abgabe 1 Autonomes Fahren}
\begin{document}
\maketitle

\tableofcontents

\section{Masse}
    \subsection{gesamtes Auto}
    Waage kann nur bis 2kg, deshalb gesplittet:
    \begin{itemize}
        \item Akku: 404,32g
        \item Fahrzeug: 1841,36g
        \item Akkuhalterung: 15,45g
        \item SUmme: 2261,13
    \end{itemize}

    \subsection{Einzelmessungen (oben drin)}
    \begin{itemize}
        \item Einzelnes Rad: 37,35g
        \item 4 Räder: 149,74g
        \item Motor: 181,87g
        \item Raspberry Pi: 50,18g
        \item IBT\_2 (blau): 65,99g
        \item Verschaltung: 48,13g
        \item Chassis: 762,99g
        \item Kameraaufhängung: 147,05g
        \item Grundplatte für Technik: 227,32g
        \item Servomotor: 63,81g
        \item Kamera: 3,38g
        \item Schalter:
        \item div Schrauben: 3,73g
        \item div Schrauben: 4,14g
        \item div Schrauben (Verbindung chassis technik): 38,11g
        \item IMU (beschleunigungssensor):
        \item Kabel blaue Platine <-> Steuerungseinheit: 7,12g
        \item Sicherung: 34,47g
    \end{itemize}

\section{Schwerpunkt}
Gemessen mit Federwagen in X Richtung.
\begin{itemize}
\item vorne: $7,1N$
\item hinten: $14,5N$
\item Abstand: $31,5cm$
\item $7,1/(7,1+14,5)*31,5 = 10,354166667$
\item $14,5/(7,1+14,5)*31,5 = 21,145833333$
\end{itemize}

Waage:

\begin{itemize}
\item vorne: 907,4g
\item rechts: 1305,3g
\item 907,4/(907,4+1305,3)*30=12,302616713
\item 1305,3/(907,4+1305,3)*30=17,697383287
\end{itemize}

Korrektur: 28,5 (TODO)

Links/Rechts:
links 11,1N rechts 9N
links 10N rechts 10,5

[0] Den wahren Schwerpunkt - irgendwo im Inneren der Karosserie (Fahrgastzelle) zu ermitteln, dürfte schwierig werden.
Soweit ich das richtig verstehe, wäre Dir aber vermutlich schon ausreichend damit geholfen, die Schwerpunktlage bezogen auf die Grundfläche zu ermitteln.

[0]: http://www.uni-protokolle.de/foren/viewt/262158,0.html
*


\section{Trägheit}
[1]: file:///Users/nico.albers/Downloads/MDO3000-Oscilloscope-User-Manual.pdf


$T=\frac{2\pi}{\omega} = 2\pi\sqrt{\frac{I}{mgl}}$
* T Periodendauer
* I Trägheitsmoment
* $\omega$: Kreisfrequenz
* m Masse
* g Erdbeschleunigung
* l Länge des Pendels



\section{Radius}
maximaler Einschlagwinkel (300), 133,7m


\section{dynamischer radius}
1. run 500
2. run 1000
3. run 2000









\end{document}
